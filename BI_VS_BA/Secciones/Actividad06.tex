\section{Conclusiones} 

Business intelligence nos permite visualizar y hacer análisis en base a la información histórica,  es decir hacia el pasado nos brinda tendencia de como se ha desarrollado historicamente el negocio de una compañia, pero hoy en dia eso no lo es todo ya que se necesita tambien analizar el futuro y eso es lo que se llama Business analytics nos permite  poder predecir en base a la experiencia en base a la información que se tiene,  en base a  modelos predictivo y anticiparse a lo que otras compañias  desconocen, es la gran diferencia una mira la pasado y otra al futuro con modelos predictivos.
\newline
\newline
\newline
\textbf{Webgrafía}\\
\newline
- https://www.betterbuys.com/bi/business-intelligence-vs-business-analytics/\\
- https://www.threepoints.com/int/executive-en-business-artificial-intelligence/\\
- https://www.sisense.com/blog/whats-the-difference-between-business-intelligence-and-business-analytics/\\
- https://www.logianalytics.com/bi-trends/business-intelligence-vs-analytics-whats-the-difference/
